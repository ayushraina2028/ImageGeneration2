\section*{NADE}

\begin{frame}{Neural Autoregressive Density Estimator(NADE)}
    Here we add a neural network layer to approximate the factors $p(X_i|X_{<i})$ \\

    \textbf{Output: } $n$ dimensional vector $p(X_i|X_{<i})$ for $i=1,2,...,D$ \\

    \bigskip

    But the $k$th output should see inputs $X_{<k}$ only. \\
\end{frame}

\begin{frame}{Adding a neural network layer}

    Consider $X \in \mathbb{R}^N$ and $h_k \in \mathbb{R}^D$ be the hidden representation for the $k$th output. \\
    \bigskip

    For the $k$th output, the hidden representation will be computed using previous $k-1$ inputs. \\

    \[ \boxed{h_k = \sigma(W_{.,<k}X_{<k} + b)} \]

    where $W$ is the weight matrix which is shared during the computation of $h_k$ for all $k$. \\

    \bigskip

    Here $W_{.,<k}$ represents first $k-1$ columns of $W$.
\end{frame}

\begin{frame}{Computing output from hidden representation}
    We now compute the output $p(X_k|X_{<k})$ using the hidden representation $h_k$ as follows: 
    
    \bigskip

    \[ y_k = p(X_k | h_k) = \sigma(V_kh_k + c_k) \]
    
\end{frame}

\begin{frame}{Final Equations}
    So here is our model: \\
    
    \[ h_k = \sigma(W_{.,<k}X_{<k} + b) \]
    \[ y_k = p(X_k | h_k) = \sigma(V_kh_k + c_k) \]

    $\Sigma = \{ W, V, b, c \}$ are the parameters of the model. \\

\end{frame}

\begin{frame}{Calculating the number of parameters}
    
    But how many parameters are there:

    \begin{enumerate}
        \item $W \in \mathbb{R}^{D \times N} \implies DN $ parameters
        \item $b = \{b_1,b_2,...,b_D\} \implies D$ parameters
        \item $V_k \in \mathbb{R}^{D}$ for $k=1,2,...,N \implies DN$ parameters
        \item $c = \{c_1,c_2,...,c_N\} \implies N$ parameters
        \item $h_1 \in \mathbb{R}^D$ is also a parameter $\implies D$ parameters
    \end{enumerate}

    \bigskip
    
    Hence total number of parameters are $2DN + 2D + N \sim O(DN)$
    
\end{frame}

\begin{frame}{Generation: Binary Random Variable}

    1. Compute $p(X_1=1) = \sigma(V_1h_1 + c_1) = t_1$ where $t_1 \in [0,1]$ \\
    2. Sample $m_1 \sim Unif[0,1]$, if $m_1 < t_1$ then $X_1 = 1$ else $X_1 = 0$ \\
    3. Compute $p(X_2=1|X_1) = \sigma(V_2h_2 + c_2) = t_2$ where $t_2 \in [0,1]$ \\
    4. Sample $m_2 \sim Unif[0,1]$, if $m_2 < t_2$ then $X_2 = 1$ else $X_2 = 0$ \\

    \bigskip

    Repeat this process for $X_3,X_4,...,X_D$ to generate a sample from the distribution $p(X)$

\end{frame}

\begin{frame}{Experimental Results}
    
\end{frame}
